\section{Conclusion}\label{sec:conclusion}
In this paper, we presented the first genuine approach to forecast a process model as a whole. To this end, we developed a technique based on time series analysis of DF relations to forecast entire DFGs from historical event data. 
%In this way, we are able to capture process drifts in an accurate way, as demonstrated in the evaluation using our Process Change Exploration system. 
In this way, we are able to make accurate forecasts regarding the future development of the process, including whether process drifts or major changes might occur in particular parts of the process. 
The forecasting approach is supported by the Process Change Exploration system which allows analysts to compare various parts of the past, present, and forecasted future behaviour.
%While further investigation is certainly required, 
Our empirical evaluation demonstrates that, most notably for reduced process models with medium-sized alphabets, we can obtain below 15\% MAPE in terms of conformance to the true models.

In future research, we plan to evaluate the use of machine learning techniques for process model forecasting. Furthermore, we want to explore opportunities for enriching our forecasted process models with confidence intervals. Finally, we will conduct design studies with process analysts to evaluate the usability of different visualisation techniques.