\documentclass{svproc}

\usepackage{url}
\usepackage{mdframed}
\usepackage{latexsym}
\usepackage{python}
\usepackage{wrapfig}
\usepackage{epstopdf}
\usepackage{array}
\usepackage{multirow}
\usepackage{caption}
\usepackage{wrapfig}
\usepackage{booktabs}

\usepackage{subfigure}
\usepackage{amssymb}
\usepackage{graphicx}
\usepackage{amssymb}
\usepackage{amsfonts}
\usepackage{bigstrut}

\usepackage[cmex10]{amsmath}
\usepackage{pgfplots}
\usepackage{algorithm}
\usepackage[noend]{algpseudocode}
\usepackage{footmisc}
\captionsetup{compatibility=false}
\usepackage{rotating}
\usepackage{xparse}
\usepackage{todonotes}
\usepackage{comment}
\usepackage{enumitem}
\usepackage{marginnote}
\usepackage{xcolor}
\usepackage{bigstrut}

%\usepackage{tabulary}
\usepackage{colortbl}

\def\UrlFont{\rmfamily}


%
% This one allows us to have a custom DESCRIPTION environment, in which numbers are preceded by a one-letter prefix.
% A short description can be added as an additional parameter to the begin-environment command.
% Successive references will have that very one-letter prefix + number with hyper-reference to it.
%
\RequirePackage{enumitem}
%
\def\requiprefix{T}
%
\newcounter{requicount}
\newlist{requidescr}{description}{1}
\setlist[requidescr,1]{%
	before={\setcounter{requicount}{0}%
		\renewcommand*\therequicount{\arabic{requicount}}},
	,font=\bfseries{\requiprefix}\stepcounter{requicount}\therequicount\normalfont.~\bfseries
}
%
% From https://tex.stackexchange.com/questions/1230/reference-name-of-description-list-item-in-latex
\makeatletter
\def\namedlabel#1{\begingroup
	\def\@currentlabel{\requiprefix\therequicount}%
	\phantomsection\label{#1}\endgroup
}
\makeatother
%
% Usage example:
% \begin{requidescr} % CUSTOM from CDC, with love
%   \item[\namedlabel{req:xaxis} Cluster on Y axis] Something more about that here.
% \end{requidescr}
% I especially like \ref{req:xaxis}.

\begin{document}
\mainmatter      
%
\title{From Time Series to Process Model Forecasting}
%
\titlerunning{Process Model Forecasting Using Time Series} 
%
\author{Johannes~De~Smedt\inst{1} \and Anton~Yeshchenko\inst{2} \and Artem~Polyvyanyy\inst{3} \and Jochen~De~Weerdt\inst{1} \and Jan~Mendling\inst{2}}
%
%
\institute{KU Leuven, Leuven, Belgium\\
\email{\{johannes.desmedt;jochen.deweerdt\}@kuleuven.be}\\
	WU Vienna, Vienna, Austria\\
\email{\{anton.yeshchenko;jan.mendling\}@wu.ac.at}
\and
University of Melbourne, Melbourne, Australia\\
\email{artem.polyvyanyy@unimelb.edu.au}
}
\maketitle              

\begin{abstract}
The surge in event-based data recorded during the execution of business processes is ever-growing and has spurred an array of process analytics techniques to support and improve information systems.
A major strand of process analytics encompasses forecasting the process's future development, mostly focusing on next-step, remaining time, or goal-oriented predictions.
The granularity of such approaches lies with the events in the process.
This work approaches process forecasting at the level of the entire process model.
This allows process analysts to act on predicted global and longer-term changes such as impending drifts in the process model rather than fine-granular ones such as next-step prediction.
To this purpose, event data is captured at various intervals and aggregated in directly-follows graphs.
The relation of each activity pair in the graph is monitored over these intervals, and their future values forecasted using standard time-series techniques.
Experiments confirm that this approach is already well-capable of informing process analysts about the future status of the process.
\keywords{Process model forecasting, predictive process modelling, process mining, time series analysis}
\end{abstract}
%
\section{Introduction}\label{sec:introduction}
Process analytics is an area of process mining~\cite{van2016data} focusing on predictions for individual process instances or overall process models. At the instance level, 
various novel techniques of predictive process monitoring (PPM)  have been recently devised, tackling problems such as next activity, remaining cycle time, or other outcome predictions~\cite{DBLP:conf/bpm/Francescomarino18}. These techniques make use of neural networks \cite{DBLP:conf/caise/TaxVRD17}, stochastic Petri nets \cite{DBLP:conf/icsoc/Rogge-SoltiW13}, and general classification techniques \cite{DBLP:journals/tkdd/TeinemaaDRM19}.

At the model level, there is a notable void. Many analytical tasks require not only an understanding of the current as-is, but also the anticipated will-be process model. A key challenge in this context is the consideration of evolution as processes are known to be subject to drift~\cite{maaradji2017detecting,DBLP:conf/bpm/PollPRRR18,yeshchenko2019comprehensive,yeshchenko2021visual}. A forecast can then inform the process analyst how the will-be process model might differ from the current as-is if no measures are taken, e.g., against emerging bottlenecks.

This paper presents the first technique to forecast whole process models. To this end, we develop a technique that builds on a representation of event data as multiple time series. Each of these time series captures the evolution of a behavioural aspect of the process model in the form of directly-follows relations (DFs), such that corresponding forecasting techniques can be applied for directly-follows graphs (DFGs). Our implementation on six real-life event logs demonstrates that forecasted models with medium-sized alphabets (10-30 activities) obtain below 10\% mean average percentage error in terms of conformance.
Furthermore, we introduce the Process Change Exploration (PCE) system which allows to visualise past and present models from event logs and compare them with forecasted models.

%The growth in the use of information systems has fuelled a wide range of data analysis techniques that intend to describe and improve their inner workings. Process mining~\cite{van2016data} is one of the most fast-growing fields in information systems analysis and encompasses the wide range of techniques performed on event data generated by these systems, including the visualisation, conformance checking, and enhancement of process models generating these events. More recently, predictive process analysis techniques, often referred to as predictive process monitoring (PPM), have surfaced to support the prediction of the next activity in the process, the remaining process cycle time, or other goal-oriented process outcomes~\cite{DBLP:conf/bpm/Francescomarino18}. PPM techniques make use of various analytics architectures, including neural networks \cite{DBLP:conf/caise/TaxVRD17}, stochastic Petri nets \cite{DBLP:conf/icsoc/Rogge-SoltiW13}, and general classification techniques \cite{DBLP:journals/tkdd/TeinemaaDRM19}.

%However, PPM techniques typically focus on a short time horizon or a well-scoped outcome in terms of prediction~\cite{DBLP:journals/tkdd/TeinemaaDRM19}. Indeed, it is known that next activity prediction models often perform poorly for long time horizons~\cite{park2020predicting}. This, consequently, limits the range of insights that can be obtained.  A process (model) is ever-changing. Hence, while the outcome can be predicted accurately, this might still obfuscate the underlying drivers for that outcome. Process analysts could benefit from obtaining a more evolutionary image of the process~\cite{DBLP:conf/bpm/PollPRRR18}, including the stability or change of (parts of) the process model such as process drifts, which can inform improvement ideas~\cite{van2015pm}. 

%This paper suggests a paradigm shift from typical PPM techniques that focus on next activity or individual outcome prediction, to predictive techniques that focus on the entire process model.
%Current predictive techniques construct a predictive model to infer the future development of a trace, while the proposed forecasting approach infers the future development of a model which can generate future traces itself.
%This is achieved via an aggregation, both at the input and output of forecasting, of directly-follows (DF) relations over all the process activity pairs as a summary of the process model over various time intervals, which are further extrapolated into the future using time series techniques modelling the number of DF occurrences over time.
%Given that forecasting over long horizons results in a vast amount of information, inspired by the ideas in~\cite{DBLP:conf/er/KabicherKR11}, we propose an approach for tracking the changes in forecasted models and comparing them with AS-IS models.
%The idea hence marries the concept of traditional process discovery resulting in descriptive models with a typical machine learning approach, which uses training and test sets in order to infer generalising models that can make quality predictions of the future. 

%This paper makes four contributions:
%\begin{enumerate}
%	\item We propose the first process model forecasting technique grounded in the time series forecasting;
%	\item We evaluate the quality of the process models forecasted by our technique using standard conformance checking measures from process mining;
%	\item We argue that the quality of the produced forecasts are mostly in the range of 5-15\% in terms of mean absolute percentage error in terms of conformance with actual models, which makes process model forecasting useful for practical applications; and
%	\item the Process Change Exploration (PCE) system is presented for comparing AS-IS and forecasted process models.
%\end{enumerate}
%To this purpose, a variety of time series analysis techniques are applied to six industrial event logs using two aggregations that establish suitable time series. 
%A variety of considerations regarding the use of time series analysis for event logs, and the suitability of other predictive algorithms, are discussed.
%Results show that simple forecasting techniques often perform adequately without extensive parameter tuning, leading to forecasted process models that are close to the actual one. Moreover, we demonstrate the usefulness of the PCE system to visually explore forecasted process models. 

This paper is structured as follows. Section \ref{sec:2:motivation} discusses related work and motivates our work. Section \ref{sec:methodology} specifies our process model forecasting technique together with the PCE visualisation environment. Section \ref{sec:experiment} describes our evaluation, before Section \ref{sec:conclusion} concludes the paper.
\section{Preliminaries}\label{sec:preliminaries}

An event log $L$ contains the recording of traces $\sigma \in L$ produced by an information system during its execution and contains a sequence of events.
Events in these traces are part of the power set over the alphabet of activities $\Sigma$ which exist in the information system $\langle e_1,...,e_{|\sigma|}\rangle \subseteq \Sigma^*$.
Directly follows relations between activities in an event log can be expressed as a counting function over activity pairs $>_L: \Sigma\times\Sigma \to \mathbb{N}$ with $>_L(a_1,a_2) = |\{e_n=a_1,e_{n+1}=a_2, \,\forall e_i\in L\}|$.
Directly follows (DF) relations can be calculated on traces and subtraces in a similar fashion.
A Directly Follows Graph (DFG) of the process then exists as the weighted directed graph with the activities as nodes and their DF relations as weights $DFG=(\Sigma,>_L)$.

In order to obtain predictions regarding the evolution of the DFG we construct DFGs for subsets of the log.
Many aggregations and bucketing techniques exist for next-step and goal-oriented outcome prediction \cite{DBLP:conf/caise/TaxVRD17,DBLP:journals/tkdd/TeinemaaDRM19}, e.g., predictions at a point in the process rely on prefixes of a certain length, or particular state aggregations \cite{DBLP:journals/sosym/AalstRVDKG10}.
In the proposed forecasting approach, however, not cross-sectional but time series data will be used.
Hence, the evolution of the DFGs will be monitored over intervals of the log where multiple aggregations are possible:
\begin{itemize}
	\item Equitemporal aggregation: each sublog contains a part of the event log of equal time duration. This can lead to sparsely populated sublogs when the events' occurrences are not uniformly spread over time, however, is easy to apply (on new traces).
	\item Equisized aggregation: each sublog contains a part of the event log of similar DF sum. This leads to well-populated sublogs, however, might be harder to apply when new data does not contain sufficient new DF occurrences.
\end{itemize}
Time series can be obtained for all $>_{Ls},\, Ls\subseteq L$ by applying the aforementioned aggregations.
Tables \ref{tab:eventlog} and \ref{tab:aggregation} provide an example of both.

\begin{table}[htbp]
   \begin{minipage}{.5\textwidth}
   	\centering
    \begin{tabular}{|l|l|l|}
    \toprule
    {Case ID} & Activity &Timestamp \\
    \midrule
    1     & A1    & 11:30 \\
    1     & A2    & 11:45 \\
    1     & A1    & 12:10 \\
    1     & A2    & 12:15 \\
    \midrule
    2     & A1    & 11:40 \\
    2     & A1    & 11:55 \\
    \midrule
    3     & A1    & 12:20 \\
    3     & A2    & 12:40 \\
    3     & A2    & 12:45 \\
    \bottomrule
    \end{tabular}
    \caption{Example event log with 3 traces and 2 activities.}
\label{tab:eventlog}%
\end{minipage}
  \begin{minipage}{.5\textwidth}
  	\centering
    \begin{tabular}{|l|c|c|}
    \toprule
    DF    & Equitemporal  & Equisized \\
    \midrule
    $<_{Ls}(A1,A1)$ & (0,1,0) & (1,0,0) \\
    $<_{Ls}(A1,A2)$ & (1,1,1) & (1,1,1) \\
    $<_{Ls}(A2,A1)$ & (0,1,0) & (0,1,0) \\
    $<_{Ls}(A2,A2)$ & (0,0,1) & (0,0,1) \\
    \bottomrule
    \end{tabular}
  \caption{An example of using an interval of 3 used for equitemporal aggregation (75 minutes in 3 intervals of 25 minutes) and equisized intervals of size 2 (6 DFs over 3 intervals)).}
  \label{tab:aggregation}
 \end{minipage}%
\end{table}%

\section{Process model forecasting}\label{sec:methodology}
This section outlines how directly-follows time series are extracted from event logs as well as how they are used to obtain process model forecasts with a range of widely-used forecasting techniques.
Finally, the visualisation of such predictions is introduced.

\input{3a-dftimeseries}
\subsection{Time series techniques}\label{sec:3b:timeseries}

To model the time series of DFs, various algorithms can be used.
In time series modelling, the main objective is to obtain a forecast or prediction $\hat{y}_{T+h|T}$ for a horizon $h\in \mathbb{N}$ based on previous $T$ values in the series $(y_1,...,y_T)$ \cite{hyndman2018forecasting}.
For example, the naive forecast simply uses the last value of the time series $T$ as its prediction $\hat{y}_{T+h|T}=y_T$.
An alternative naive forecast uses the average value of the time series $T$ as its prediction $\hat{y}_{T+h|T}=\frac{1}{T}\Sigma_i^{T} y_i$.
A wide array of such forecasting techniques exist, ranging from simple models such as naive forecasts over to more advanced approaches such as exponential smoothing and auto-regressive models.
Many also exist in a seasonal variant due to their application in contexts such as sales forecasting.
We briefly discuss smoothing models, autoregressive, moving averages and ARIMA models, and varying variance models which make up the main families of traditional time series forecasting \cite{hyndman2018forecasting}.

A Simple Exponential Smoothing (SES) model uses a weighted average of past values where their importance exponentially decays as they are further into the past where Holt's models introduce a trend in the forecast, meaning the forecast is not flat.
Exponential smoothing models often perform very well despite their simple setup \cite{makridakis2018statistical}.

AutoRegressive Integrating Moving Average (ARIMA) models are based on auto-correlations within time series. 
They combine auto-regressions with a moving average over error terms.
It is established by a combination of an AutoRegressive (AR) model of order $p$ uses the past $p$ values in the time series and applies a regression over them and a Moving Average (MA) model of order $q$ which creates a moving average of the past forecast errors.
Given the necessity of using a white noise series for AR and MA models, data is often differenced to obtain such series.
ARIMA models then combine both AR and MA models where the integration is taking place after modelling as these models are fitted over differenced time series.
ARIMA models are considered to be one of the strongest time series modelling techniques.

An extension to ARIMA which is widely used in econometrics exists in (Generalized) AutoRegressive Conditional Heteroskedasticity ((G)ARCH) models \cite{francq2019garch}.
They resolve the assumption that the variance of the error term has to be equal over time, but rather model this variance as a function of the previous error term.
For AR-models, this leads to the use of ARCH-models, while for ARMA models GARCH-models are used as follows.
An ARCH(q) model captures the change in variance by allowing it to both gradually increase over time, or to allow for short bursts of increased variance.
A GARCH(p,q) model combines both the past values of observations as well as the past values of variance.
(G)ARCH models often outperform ARIMA models in contexts such as the prediction of financial indicators of which the variance often changes over time \cite{francq2019garch}.
\subsection{Process change exploration}\label{sec:3c:pce}

In this section...



\section{Implementation and evaluation}\label{sec:experiment}
In this section, an experimental evaluation over six real-life event logs is reported.
The aim of the evaluation is to measure to what extent the forecasted process models/DFGs are capable of correctly reproducing actual future DFGs in terms of allowing for the same process model behaviour.
To this purpose we benchmark the actual against the forecasted entropic relevance discussed in \Cref{sec:2:motivation}.
This is done at various parts of the trace, i.e. forecasts for the middle of the event logs up to the later parts of the event log to capture the robustness of the forecasting techniques in terms of the amount of data required to obtain good prediction results for both the equisized and equitemporal aggregation.

\subsection{Re-sampling and test setup}
To obtain training data, time series are obtained by specifying a number of intervals (i.e. time steps in the DF time series) using either equitemporal or equisized aggregation a described in Section \ref{sec:3a:preliminaries}.
Time series algorithms are parametric and sensitive to sample size requirements \cite{hanke2001business}.
Depending on the number of parameters a model uses, a minimum size of at least 50 steps is not uncommon, although typically model performance should be monitored at a varying number of steps.
In the experimental evaluation, the event logs are divided into 100 time intervals with a varying share of training and test intervals. A constant and long horizon $h=25$ is used meaning all test sets contain 25 intervals, but the training sets are varied from $ts=25$ to $ts=75$ intervals, meaning the forecasts progressively target the prediction of intervals 25-50 (the second quarter of intervals) over to 75-100 (the last quarter of intervals).
This allows to both inspect the difference in results when only few data points are used, and whether there is a difference forecasting data points in the middle or towards the end of the available event data.

Resampling is applied based on a 10-fold cross-validation constructed following a rolling window approach for all horizon values $h\in[1,25]$ where a recursive strategy is used to iteratively obtain $\hat{y}_{t+h|T_{t+h-1}}$ with $(y_1,\dots,y_{T},\dots,\hat{y}_{t+h-1})$ \cite{weigend2018time}.
10 training sets are hence constructed for each training set length $ts$ and exist from $(y_1,\dots,y_{T-h-f})$ and the test sets from $(y_{T-h-f+1},\dots,y_{T-f})$ with $f\in[0,9]$ the fold index \cite{bergmeir2012use}.
While direct strategies with a separate model for every value of $h$ can be used as well and avoid the accumulation of error, they do not take into account statistical dependencies for subsequent predictions.

Three popular publicly-available event logs are used: the BPI challenge of 2012 log\footnote{\url{https://doi.org/10.4121/uuid:3926db30-f712-4394-aebc-75976070e91f}}, 2017\footnote{\url{https://doi.org/10.4121/uuid:5f3067df-f10b-45da-b98b-86ae4c7a310b}}, and 2018\footnote{\url{https://doi.org/10.4121/uuid:3301445f-95e8-4ff0-98a4-901f1f204972}}, the Sepsis cases event log\footnote{\url{https://doi.org/10.4121/uuid:915d2bfb-7e84-49ad-a286-dc35f063a460}}, an Italian help desk log\footnote{\url{https://doi.org/10.4121/uuid:0c60edf1-6f83-4e75-9367-4c63b3e9d5bb}}, and the Road Traffic Fine Management Process log (RTFMP) event log (see Section \ref{sec:2:motivation}).
Each of these logs has a diverse set of characteristics in terms of case and activity volume, as well as average trace length, as can be seen in Table \ref{tab:eventlogs}.
\begin{table}[htbp]
  \centering
  \resizebox{0.6\textwidth}{!}{
    \begin{tabular}{lrrr}
    \toprule
    \textbf{Event log} & \multicolumn{1}{l}{\textbf{\# cases}} & \multicolumn{1}{l}{\textbf{\# activities}} & \multicolumn{1}{l}{\textbf{Average trace length}} \\
    \midrule
    \textbf{BPI 12} & 13,087 & 36    & 20.02 \\
    \textbf{BPI 17} & 31,509 & 26    & 36.83 \\
    \textbf{BPI 18} & 43,809 & 170   & 57.39 \\
    \textbf{Sepsis} & 1,050 & 16    & 14.49 \\
    \textbf{RTFMP} & 150,370 & 11    & 3.73 \\
    \textbf{Italian} & 4,580 & 14    & 4.66 \\
    \bottomrule
    \end{tabular}%
    }
  \caption{Overview of the characteristics of the event logs used in the experimental evaluation.}
  \label{tab:eventlogs}%
\end{table}%

%An example of applying the equisized or equitemporal aggregation to the Sepsis event log with 100 intervals results in the DF time series of Figure \ref{fig:sepsists}, where the DF occurrences of the most frequently occurring activity pair is included.
%For the equisized aggregation, the number of DFs is indeed relatively stable over the log's timeline where for the equitemporal aggregation a noticeable decline of DF pairs is visible towards the end of the series.
%This phenomenon is typical in event logs, as processes usually have particular endpoint activities, but can also be due to the unequal distribution of events over the event log's time line.
There are a few considerations concerning the DF time series in these event logs.
Firstly, DFs of activity pairs containing end point activities (i.e. at the start/end of a trace) often only contain meaningful numbers at very particular parts of the series and are hard to process by longitudinal algorithms which require a longer pattern to extract a meaningful pattern for prediction.
Secondly, especially the equitemporal aggregation can suffer from event logs in which events do not occur as frequently throughout the full log's development.
E.g., the Sepsis log's number of event occurrences tails off towards the end which can be alleviated by pre-processing (not done here to remain consistent over the event logs).
Finally, if the level of occurrences of the DF pair is low and close to 0, the series might be too unsuitable for analysis with white noise series analysis techniques that assume stationarity.
Ideally, every time series is tested using a stationarity test such as the Dickey-Fuller unit root test \cite{leybourne1995testing} and an appropriate lag order is established for differencing. 
Furthermore for each algorithm, especially ARIMA-based models, (partial) auto-correlation could establish the ideal $p$ and $q$ parameters.
However, for the sake of simplicity and to avoid solutions where each activity pair has to have different parameters, various values are used for $p$, $d$, and $q$ and applied to all DF pairs where only the best-performing are reported below for comparison with the other time series techniques.
The results contain the best-performing representative of each forecasting family.
% \begin{figure}[tb]
% 	\centering
% 	\subfigure[Most common DF - equisize]{\includegraphics[width=0.39\textwidth]{./img/rtfmp_1.png}}
% 	\subfigure[Most common DF - equitemp]{\includegraphics[width=0.39\textwidth]{./img/rtfmp_1_t.png}}
% 	\caption{RTFMP}
% 	\label{fig:sepsists}
% \end{figure}

% \subsection{Evaluation criteria}
% Given that we want to evaluate the capability of the approach to accurately predict the evolution of the process model, the combination of all DF predictions is considered to obtain a global DFG prediction.
% The following two criteria are used:
% \begin{itemize}
% 	\item \textbf{Cosine distance:} measures the distance between two vectors and is often used to compare graph distance. This metric is used to compare the DFGs' edge weight matrices between the actual and predicted number of DF relations.
% 	\item \textbf{Entropic relevance:} see section \ref{sec:2:motivation}.
% \end{itemize}
% These criteria balance a predictive and structural evaluation of the algorithms and report on both the numeric performance common in a forecasting setting as well as their appropriateness in terms of reproducing a structurally usable process model which allows for the observed process behaviour.
% In both cases a lower score is better.

\subsection{Results}\label{sec:4.3:results}
All pre-processing was done in Python with a combination of \emph{pm4py}\footnote{\url{https://pm4py.fit.fraunhofer.de}} and the \emph{statsmodels} package \cite{seabold2010statsmodels}. 
The code is available here\footnote{\url{https://github.com/JohannesDeSmedt/pmf}}.

In order to get a grasp of the forecasting performance in combination with the actual use of DFGs (which are seldom used in their non-aggregated form \cite{van2019practitioner}) we present the mean absolute percentage error (MAPE) between the entropic relevance of the actual and forecasted DFGs at both full size, at 50\%, and 75\% reduction which is node-based (i.e. only the Q2/Q3 percentile of nodes in terms of frequency is retained).
Using different levels of aggregation also balances recall and precision, as aggregated DFGs are less precise but possibly less overfitting.
The results can be found in Tables \ref{tab:result_dfg_table} to \ref{tab:result_dfg_table_25}.
NAs are reported when the algorithms did not converge, no data was available (e.g. Sepsis for the 75-100 equitemporal intervals), or extremely high values were predicted.

When no reduction is applied, Table \ref{tab:result_dfg_table} shows that for the BPI12 and 17 logs a below 10\% error can be achieved, mostly for equisized aggregation. 
For the Italian help desk log, results are in the 10-37\% bracket, while for the other logs results are often well above a 100\% deviation (with the entropic relavance of the actual DFGs being lower, hence better, than the entropic relevance of the forecasted DFGs).
For the RTFMP and BPI18 log, however, results are better when more training points are used (e.g. 50 or 75 to obtain predictions for the 50-75 and 75-100 intervals).
There is no strong difference between equisized and equitemporal aggregation except for the occasional outliers.
Overall, the percentage error is lower in Table \ref{tab:result_dfg_table_50} when a reduction of 50\% is applied with sub-10\% results for the BPI12, Sepsis, and BPI17 logs. 
The results for the RTFMP log are occassionally better, but mostly worse, similar to BPI18.
Finally, the results in Table \ref{tab:result_dfg_table_25} show a further reduction of errors for the BPI12, Sepsis, BPI17, and Italian logs, but also a drastic decrease to close to 0\% for RTFMP.
The results for the BPI18 log remain bad at over 100\% error rates.

These results are commensurate the findings in \cite{DBLP:conf/icpm/PolyvyanyyMG20} which contains entropic relevance results for the BPI12, Sepsis, and RTFMP logs indicating that entropic relevance of larger DFGs is lower (better) for RTFMP/Sespsis, and the entropic relevance goes up strongly for small models of RTFMP meaning the drastically improved error rates reported here are for models performing worse in terms of recall and precision.
The entropic relevance for the BPI12 log is stable for the full spectrum of DFG sizes as per \cite{DBLP:conf/icpm/PolyvyanyyMG20} which is reflected in the consistently good error rates presented here.
This means that the low error rates reported for the reduced DFGs which still score strongly in terms of recall and precision.
Matching all results to the event log characteristics, we notice that the event logs with longer average traces with medium-sized alphabets ($>$20) such as BPI12 and BPI17 consistently report good results.
The BPI18 log's high number of activities seems to inflate error rates quickly, which is further aggregavated when DFGs are reduced.
Given that DFGs are based on activity pairs, this result is not surprising.
For Sepsis and the Italian event logs good error rates are obtained once DFGs are reduced, indicating that forecasting the low-frequent edges and activities might lead to high error rates when the alphabet is smaller and traces are shorter, which is potentially also caused by the lack of precision as witnessed with the RTFMP log.

Overall, there exist many scenarios in which process model forecasting is delivering strong results.
For the BPI12, BPI17, Italian, and Sepsis event logs, sub-10\% error rates can be achieved both for equisized and equitemporal aggregation combined with model reductions which are typically applied by readers of DFGs.
In some cases, even a naive forecast is enough to obtain a low error rate, however, the AR and ARIMA models report the best error rates in most cases.
Nevertheless, results are often close except when fewer training points are used.
Then, results are often varying widely.
In future attempts, the confidence intervals will also be investigated.

\begin{table}[ht]
  \centering
  \resizebox{\textwidth}{!}{
    \begin{tabular}{|c|l|ccc|ccc|ccc|ccc|ccc|ccc|}
\cmidrule{3-20}    \multicolumn{1}{r}{} &       & \multicolumn{3}{c|}{\textbf{BPI 12}} & \multicolumn{3}{c|}{\textbf{Sepsis}} & \multicolumn{3}{c|}{\textbf{RTFMP}} & \multicolumn{3}{c|}{\textbf{BPI17}} & \multicolumn{3}{c|}{\textbf{Italian}} & \multicolumn{3}{c|}{\textbf{BPI18}} \\
    \multicolumn{1}{r}{} &       & \textbf{50} & \textbf{75} & \textbf{100} & \textbf{50} & \textbf{75} & \textbf{100} & \textbf{50} & \textbf{75} & \textbf{100} & \textbf{50} & \textbf{75} & \textbf{100} & \textbf{50} & \textbf{75} & \textbf{100} & \textbf{50} & \textbf{75} & \textbf{100} \\
    \midrule
    \multirow{5}[2]{*}{\begin{sideways}\textbf{equisize}\end{sideways}} & \textbf{nav} & 9.74  & 8.56  & 9.82  & 97.09 & 97.40 & 100.76 & 437.14 & \textbf{105.81} & 115.34 & 6.86  & 8.80  & \textbf{7.00} & 25.93 & 16.52 & 37.71 & 82.10 & \textbf{99.90} & 38.41 \\
          & \textbf{arima212} & 12.41 & 9.75  & 10.80 & NA    & \textbf{83.31} & \textbf{100.58} & \textbf{398.66} & NA    & NA    & 10.03 & \textbf{8.54} & 13.23 & 24.60 & 9.17  & 39.01 & 82.81 & NA    & 30.12 \\
          & \textbf{ar2} & NA    & \textbf{8.45} & \textbf{9.62} & \textbf{97.04} & 97.40 & 100.76 & NA    & NA    & \textbf{110.14} & 6.83  & 14.84 & 13.83 & 23.81 & 13.98 & \textbf{36.89} & \textbf{78.82} & NA    & NA \\
          & \textbf{hw} & \textbf{8.61} & 8.96  & 10.14 & 97.09 & 97.40 & 100.76 & 402.83 & 110.17 & 130.10 & \textbf{6.81} & 8.68  & 186.94 & \textbf{22.54} & \textbf{9.14} & 43.31 & 81.04 & NA    & NA \\
          & \textbf{garch} & 11.47 & 8.60  & 10.17 & 97.09 & 97.40 & 100.76 & 426.71 & 109.79 & 117.15 & 6.89  & 8.82  & 186.94 & 25.48 & 31.29 & 65.54 & 72.89 & NA    & \textbf{28.59} \\
    \midrule
    \multicolumn{1}{|c|}{\multirow{5}[2]{*}{\begin{sideways}\textbf{equitemp}\end{sideways}}} & \textbf{nav} & 15.57 & 10.14 & 12.63 & 98.51 & 100.75 & NA    & 199.69 & 29.70 & 36.15 & 7.12  & 8.63  & \textbf{13.41} & 27.12 & 26.86 & 39.94 & NA    & NA    & 54.57 \\
          & \textbf{arima212} & NA    & 11.67 & \textbf{12.00} & \textbf{89.07} & \textbf{100.39} & NA    & \textbf{122.63} & \textbf{28.55} & \textbf{33.82} & 8.13  & 158.70 & 18.74 & 26.59 & 24.26 & 38.03 & NA    & \textbf{42.83} & NA \\
          & \textbf{ar2} & NA    & \textbf{9.97} & 12.43 & 98.37 & 100.75 & NA    & NA    & 29.74 & NA    & 7.09  & NA    & 19.60 & \textbf{26.33} & 30.02 & 38.68 & NA    & NA    & NA \\
          & \textbf{hw} & \textbf{13.09} & 10.46 & 12.08 & 98.40 & 100.75 & NA    & 162.94 & 29.34 & 36.15 & \textbf{7.07} & \textbf{8.35} & 186.91 & 26.90 & \textbf{23.57} & \textbf{36.20} & NA    & 43.02 & NA \\
          & \textbf{garch} & 17.80 & 10.29 & 12.71 & 95.75 & 100.75 & NA    & 199.13 & 30.44 & 36.00 & 7.37  & 187.45 & 186.91 & 27.11 & 45.58 & 55.67 & NA    & 46.69 & \textbf{42.97} \\
    \bottomrule
    \end{tabular}%

    }
    \caption{Overview of the mean percentage error in terms of entropic relevance for the full DFGs.}
  \label{tab:result_dfg_table}%
\end{table}%


\begin{table}[ht]
  \centering
  \resizebox{\textwidth}{!}{
    \begin{tabular}{|c|l|ccc|ccc|ccc|ccc|ccc|ccc|}
\cmidrule{3-20}    \multicolumn{1}{r}{} &       & \multicolumn{3}{c|}{\textbf{BPI 12}} & \multicolumn{3}{c|}{\textbf{Sepsis}} & \multicolumn{3}{c|}{\textbf{RTFMP}} & \multicolumn{3}{c|}{\textbf{BPI17}} & \multicolumn{3}{c|}{\textbf{Italian}} & \multicolumn{3}{c|}{\textbf{BPI18}} \\
    \multicolumn{1}{r}{} &       & \textbf{50} & \textbf{75} & \textbf{100} & \textbf{50} & \textbf{75} & \textbf{100} & \textbf{50} & \textbf{75} & \textbf{100} & \textbf{50} & \textbf{75} & \textbf{100} & \textbf{50} & \textbf{75} & \textbf{100} & \textbf{50} & \textbf{75} & \textbf{100} \\
    \midrule
    \multirow{5}[2]{*}{\begin{sideways}\textbf{equisize}\end{sideways}} & \textbf{nav} & \textbf{4.65} & \textbf{5.83} & \textbf{11.50} & \textbf{8.35} & 8.80  & 6.29  & 234.18 & 295.99 & 203.68 & 7.82  & 9.22  & 11.04 & 23.05 & 14.18 & 21.66 & \textbf{252.76} & \textbf{231.44} & \textbf{160.66} \\
          & \textbf{arima212} & 7.96  & 22.89 & 13.43 & 8.55  & 8.81  & \textbf{6.14} & 234.14 & 288.27 & 198.86 & 4.49  & 5.98  & 10.67 & 22.31 & 7.32  & 23.17 & 369.47 & 252.24 & 218.93 \\
          & \textbf{ar2} & 24.53 & 27.58 & 30.81 & 8.54  & 8.72  & 6.30  & 234.57 & 293.10 & 201.21 & \textbf{4.27} & 6.08  & \textbf{10.22} & 21.26 & 11.91 & \textbf{20.56} & NA    & 230.02 & NA \\
          & \textbf{hw} & 45.80 & 38.02 & 13.13 & 8.73  & \textbf{8.65} & 6.17  & 233.05 & \textbf{151.19} & \textbf{111.89} & 4.51  & \textbf{5.37} & 11.06 & \textbf{20.33} & \textbf{7.28} & 26.12 & 391.38 & 280.59 & 226.27 \\
          & \textbf{garch} & 26.30 & 23.77 & 48.86 & 8.63  & 8.87  & 7.06  & \textbf{231.70} & 295.93 & 203.45 & 4.50  & 9.41  & 11.09 & 23.18 & 29.07 & 45.31 & 315.99 & 234.79 & 217.62 \\
    \midrule
    \multicolumn{1}{|c|}{\multirow{5}[2]{*}{\begin{sideways}\textbf{equitemp}\end{sideways}}} & \textbf{nav} & \textbf{7.15} & \textbf{6.86} & 17.86 & 6.41  & \textbf{7.73} & NA    & \textbf{75.48} & \textbf{36.18} & \textbf{86.46} & 5.93  & 7.23  & 30.98 & 24.35 & 18.64 & 26.48 & NA    & \textbf{219.13} & 410.13 \\
          & \textbf{arima212} & 49.87 & 10.59 & 19.59 & \textbf{4.91} & 8.13  & NA    & 135.97 & 40.22 & 86.74 & \textbf{5.64} & \textbf{5.13} & \textbf{30.30} & \textbf{23.48} & 16.32 & \textbf{20.45} & \textbf{205.21} & 261.81 & \textbf{253.38} \\
          & \textbf{ar2} & 21.06 & 7.49  & 18.85 & 7.17  & 8.08  & NA    & 95.44 & 36.30 & 86.60 & 5.70  & NA    & 30.97 & 23.67 & 21.71 & 25.44 & NA    & NA    & 443.76 \\
          & \textbf{hw} & 7.41  & 7.02  & \textbf{17.54} & 6.72  & 10.34 & NA    & 77.93 & 36.62 & 86.76 & 5.95  & 7.39  & 30.86 & 23.55 & \textbf{15.66} & 22.91 & NA    & 236.37 & 439.04 \\
          & \textbf{garch} & 57.44 & 32.85 & 37.98 & 6.76  & 7.85  & NA    & \textbf{75.48} & 36.20 & 86.52 & 5.93  & 7.33  & 31.12 & 24.24 & 36.51 & 40.40 & NA    & 283.40 & 492.85 \\
    \bottomrule
    \end{tabular}%
    }
    \caption{Overview of the mean percentage error in terms of entropic relevance for the DFGs with a 50\% reduction.}
  \label{tab:result_dfg_table_50}%
\end{table}%


\begin{table}[ht]
  \centering
  \resizebox{\textwidth}{!}{
    \begin{tabular}{|c|l|ccc|ccc|ccc|ccc|ccc|ccc|}
\cmidrule{3-20}    \multicolumn{1}{r}{} &       & \multicolumn{3}{c|}{\textbf{BPI 12}} & \multicolumn{3}{c|}{\textbf{Sepsis}} & \multicolumn{3}{c|}{\textbf{RTFMP}} & \multicolumn{3}{c|}{\textbf{BPI17}} & \multicolumn{3}{c|}{\textbf{Italian}} & \multicolumn{3}{c|}{\textbf{BPI18}} \\
    \multicolumn{1}{r}{} &       & \textbf{50} & \textbf{75} & \textbf{100} & \textbf{50} & \textbf{75} & \textbf{100} & \textbf{50} & \textbf{75} & \textbf{100} & \textbf{50} & \textbf{75} & \textbf{100} & \textbf{50} & \textbf{75} & \textbf{100} & \textbf{50} & \textbf{75} & \textbf{100} \\
    \midrule
    \multirow{5}[2]{*}{\begin{sideways}\textbf{equisize}\end{sideways}} & \textbf{nav} & 0.96  & 0.87  & 1.37  & 2.40  & 2.91  & \textbf{3.47} & 0.00  & 0.00  & 0.01  & 0.12  & \textbf{0.29} & \textbf{0.14} & 13.11 & 15.11 & 27.02 & \textbf{247.32} & 223.34 & \textbf{146.27} \\
          & \textbf{arima212} & 0.98  & \textbf{0.84} & \textbf{1.33} & \textbf{2.38} & 2.85  & 3.65  & 0.00  & 0.00  & 0.01  & \textbf{0.06} & 0.30  & 0.16  & 13.08 & 14.94 & 26.97 & 346.94 & 236.73 & 211.17 \\
          & \textbf{ar2} & \textbf{0.86} & 0.85  & 1.35  & 2.42  & 2.58  & \textbf{3.47} & 0.00  & 0.00  & 0.01  & 0.12  & 0.30  & \textbf{0.14} & 13.10 & 15.02 & 26.97 & NA    & \textbf{222.06} & NA \\
          & \textbf{hw} & 1.00  & 0.85  & 1.35  & 2.73  & 2.90  & \textbf{3.47} & 0.00  & 0.00  & 0.01  & 0.11  & 0.31  & 0.15  & \textbf{12.98} & \textbf{14.79} & 27.08 & 333.80 & 255.83 & 207.86 \\
          & \textbf{garch} & 0.89  & 0.86  & 1.35  & 2.51  & \textbf{2.49} & \textbf{3.47} & 0.00  & 0.00  & 0.01  & 0.12  & \textbf{0.29} & \textbf{0.14} & 13.13 & 15.00 & \textbf{26.89} & 299.59 & 248.75 & 182.07 \\
    \midrule
    \multicolumn{1}{|c|}{\multirow{5}[2]{*}{\begin{sideways}\textbf{equitemp}\end{sideways}}} & \textbf{nav} & 4.92  & 3.55  & 4.05  & \textbf{2.31} & \textbf{1.65} & NA    & 0.03  & 0.02  & 0.11  & \textbf{0.05} & \textbf{0.11} & 5.92  & 8.18  & 30.10 & 20.61 & NA    & \textbf{203.92} & 562.09 \\
          & \textbf{arima212} & 4.93  & 3.59  & 3.86  & 2.77  & 2.62  & NA    & 0.03  & 0.02  & 0.11  & 0.11  & 0.14  & \textbf{5.82} & 8.39  & 30.20 & 20.54 & \textbf{180.36} & 245.18 & \textbf{191.25} \\
          & \textbf{ar2} & 4.85  & \textbf{3.52} & 4.04  & 2.35  & 2.93  & NA    & 0.03  & 0.02  & 0.11  & 0.06  & NA    & 5.91  & 8.19  & 30.17 & 20.58 & NA    & NA    & 559.14 \\
          & \textbf{hw} & \textbf{4.82} & \textbf{3.52} & 3.84  & 2.47  & 1.67  & NA    & 0.03  & 0.02  & 0.11  & 0.06  & 0.13  & 5.85  & 8.45  & \textbf{20.00} & \textbf{15.90} & NA    & 228.78 & 384.37 \\
          & \textbf{garch} & 7.97  & 3.54  & 4.02  & 2.34  & 2.92  & NA    & 0.03  & 0.02  & 0.11  & \textbf{0.05} & 0.12  & 5.93  & \textbf{8.17} & 30.03 & 20.52 & NA    & 226.58 & 606.03 \\
    \bottomrule
    \end{tabular}%
    }
    \caption{Overview of the mean percentage error in terms of entropic relevance for the DFGs with a 75\% reduction.}
  \label{tab:result_dfg_table_25}%
\end{table}%




% \begin{figure}
%     \centering
%     \includegraphics[width=0.8\textwidth]{img/cv_entropic_small_equisize.png}
%     % \includegraphics[width=0.8\textwidth]{img/cv_bpi12_entropic_small_equisize.png}
%     % \includegraphics[width=0.8\textwidth]{img/cv_sepsis_entropic_small_equisize.png}
%     % \includegraphics[width=0.8\textwidth]{img/cv_rtfmp_entropic_small_equisize.png}
%     \caption{Entropic relevance results for equisize aggregation. Best (lowest) results are indicated in bold.}
%     \label{fig:equisize}
% \end{figure}
% \begin{figure}
%     \centering
%     \includegraphics[width=0.8\textwidth]{img/cv_entropic_small_equitemp.png}
%     % \includegraphics[width=0.8\textwidth]{img/cv_bpi12_entropic_small_equitemp.png}
%     % \includegraphics[width=0.8\textwidth]{img/cv_sepsis_entropic_small_equitemp.png}
%     % \includegraphics[width=0.8\textwidth]{img/cv_rtfmp_entropic_small_equitemp.png}
%     \caption{Entropic relevance results for equitemporal aggregation. Best (lowest) results are indicated in bold.}
%     \label{fig:equitemp}
% \end{figure}

% \subsection{Managerial implications}
% Depending on the event log, it is feasible to obtain good (\textless15\%) to very good (\textless5\%) forecasting results in terms of the MAPE and entropic relevance against actual DFGs.
% The experiments show that the number of intervals used for training vastly impacts results, which is expected given that with 10-fold cross-validation, this results in only 15 interval being used for the tenth fold to predict the next 25 intervals.
% Given that GARCH models often perform strongly suggests that DF time series benefit from the use of models which allow for a varying level of variance.
% This makes sense in an event log context, as activities generating the DF occurrences do not occur uniformly during the execution of a process.

\vspace{-1.5cm}
\subsection{Visualising Process Model Forecasts}\label{sec:visualisation}

In~\Cref{sec:4.3:results}, we evaluated forecasting results, ensuring the conformance and interpretability of the predicted process models. To that end, gaining insights from such predicted data remains a difficult task for the analyst. 
This section sets off to present a novel visualisation system to aid analysts in exploring the event logs. The process of designing and implementing the system started by designing several prototypes that undergone rounds of discussions to mature into the implemented visualisation system. 

The design of the PCE system is shown in Figure~\ref{fig:vis-two-brushes}. It offers an interactive visualisation system with several connected views. The system is implemented using the D3.js JavaScript library and is available as an open-source project\footnote{See \url{https://github.com/yesanton/Process-Change-Exploration-Visualizations}}.

\begin{figure}
	\centering
	\includegraphics[width=\textwidth]{img/vis/actual-predicted-two-brushed-regions-system.PNG}
	\caption{Process Change Exploration (PCE) system.~\emph{(a)} shows~\emph{Adaptation Directly-Follows Graph (aDFG)} view.~\emph{(b)} shows the \emph{Timeline view with brushed regions} view. Users can brush one or more regions on this graph in order to filter the scope of the analysis~\emph{(b.1}, and~\emph{b.2)}. Two additional views on~\emph{(c)} show the \emph{activity and path sliders}.} 
	\label{fig:vis-two-brushes}
\end{figure}
\section{Visualising Process Model Forecasts}\label{sec:visualisation}


% as has been shown the prediction is nice but it is difficult to derive the insignts 
% definition of the visualization
% humans cannot comprehend the results and we need visualzation 
% (new functions on vis that ar enot mentioned before should be mentioned?)

%%%
% outline:
% 1. requirements
% 2. design (and what supports design)
% 3. system overview
%%%
The evaluation Section~\ref{sec:experiment} has shown a good performance of the prediction algorithms for the process model forecasts. To that end, deriving insights from such predicted data remains a difficult task for the analyst. In this section we present the design of visualization system that aids analysts in exploration of the past and future of the processes.

We designed a Process Change Exploration (PCE) system to support the interpretation of the process model forecasts. In order to design the system we first established requirements for the visualization based on the related literature and experience working with event sequence data. We then designed several prototypes, that after several rounds of discussions matured into the implemented visualization system.

To derive the requirements we focus on the requirements of process mining analysis with respect to process forecasting and visualization principles. The authors of~\cite{DBLP:conf/bpm/PollPRRR18} discuss the opportunities for process forecasting. They describe that the possible utility of the process forecasting is an understanding the incremental changes or adaptations that happen to the process model into the future. In designing an exploitative visualization system, we followed the "Visual Information-Seeking Mantra:"~\emph{overview first, zoom and filter, then details-on-demand}~\cite{DBLP:conf/vl/Shneiderman96}. 
(maybe talk about tasks? not requirements)
Thus, we expect our system to assist in:

\begin{requidescr}
	\item[Identify process adaptations:\namedlabel{req:adaptation}] The visualisation system should assist the user in identifying the changes changes that happen in the process model;
	\item[Allow for interactive exploration:\namedlabel{req:interactive}] The user should be able to follow the visual information-seeking principles;
	
\end{requidescr} % CUSTOM from CDC, with love :)





\begin{figure}
	\centering
	\includegraphics[width=\textwidth]{img/vis/vis-system-two-brushes.png}
	\caption{Process Change Exploration (PCE) system. The interactive system consists of three parts. Part b shows the timeline of the data (green region), and forecasted values (grey region). The user can brush one or two regions (in this b.1 and b.2) on this graph to see the filtered for that time range process model or a difference between two process models. The main view (a) represents the directly follows change graph that shows the difference between two brushed regions. The c.1 and c.2 are the usual filters on the number of paths and activities to aid simplification the visual representation} 
	\label{fig:vis-two-brushes}
\end{figure}

\begin{figure}
	\centering
	\includegraphics[width=\textwidth]{img/vis/vis-system-one-brush.png}
	\caption{Process Change Exploration (PCE) system. The interactive system consists of three parts. Part b shows the timeline of the data (green region), and forecasted values (grey region). The user can brush one time range (in this b.0) on this graph to see the filtered for that time range process model. The main view (a) represents the directly follows graph of that region. The c.1 and c.2 are the usual filters on the number of paths and activities to aid simplification the visual representation} 
	\label{fig:vis-one-brushes}
\end{figure}


\noindent\textbf{%
	User interface
} The Figure~\ref{fig:vis-two-brushes} displays the screenshot of the PCE visualization system. 
We improve upon the notion of Directly-Follows graph~\cite{leemans2019directly} that is widely used in process mining research and practice. We use the ideas from the version graph~\cite{DBLP:conf/grapp/KriglsteinR12} on how to represent the change between versions of the graph with coloring of the transitions. 

..... and so on

The interactive visualization system was implemented in D3.JS JavaScript visualization library.










\section{Conclusion}\label{sec:conclusion}
In this paper we investigated the potential of using time series forecasting for process model monitoring.

In future research we will cover more intricate forecasting techniques and compare with machine learning-based models.
Furthermore, we will perform an extensive prediction confidence interval analysis.

\bibliographystyle{splncs03}
\bibliography{lib}


\end{document}
