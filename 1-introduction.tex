\section{Introduction}\label{sec:introduction}

Process mining...

Various predictive techniques exist in the context of process mining, many coined under the term of predictive process monitoring [\cite{DBLP:conf/caise/TaxVRD17,DBLP:journals/tkdd/TeinemaaDRM19}...]...


The approach presented in this paper entails a paradigm shift from these typical predictive applications which span remaining time, next-step, or goal-oriented prediction \cite{DBLP:conf/bpm/Francescomarino18} as the focus is not on the level of single outcomes at the level of the trace and/or event but on the model.
These predictive techniques construct a predictive model to infer the future development of a trace, while the proposed forecasting approach infers a model which can generate future traces itself.
This is achieved by an aggregation at the model level both at the input as well as the output of the learning process.
This is done by using directly-follows relations over all activity pairs as a summary of the process model at various time steps, which are further extrapolated into the future using time series techniques.
An initial attempt to set up process model forecasting was presented in \cite{de2020predictive}, however, the level of granularity of the forecasted horizons is coarser compared to the higher-frequency intervals used for time series underpinning the proposed approach.

Where process mining focuses on learning the as-is model to inform the to-be model and suggest potential repairs and improvements, process model forecasting allows to already grasp the future outcomes of the current as-is process which allows to shortcut potentially wrong outcomes \cite{DBLP:conf/bpm/PollPRRR18}.

Next to the forecasting aspect, we also present how to visualize the forecasted models and show how they exist of adaptations to the current as-is model with a level of uncertainty through change graphs \cite{DBLP:conf/er/KabicherKR11}.

This paper makes three contributions:
\begin{enumerate}
	\item Proposes the first process model forecasting technique, which fits the the broader area of predictive process monitoring;
	\item Evaluates the quality of the forecasted process models using standard conformance checking techniques from process mining;
	\item Confirms that the achieved quality of forecasts is useful for practical applications.
\end{enumerate}

To this purpose, a variety of time series analysis techniques are applied to 3 real-life event logs with two event log aggregations that establish suitable time series. They predict the level of directly-follows occurrences for activity pairs over time.
A variety of considerations regarding the use of time series analysis for event logs are discussed, as well as the suitability of other predictive algorithms.
Results show that simple forecasting techniques often already perform adequately without extensive parameter tuning. More intricate models often fail to report consistent performance.

This paper is structured as follows...