\section{Introduction}\label{sec:introduction}

The growth in the use of information systems has fuelled a wide range of data analysis techniques that intend to describe and improve their inner workings.
Process mining~\cite{van2016data} is one of the most fast-growing fields in information systems analysis and encompasses the wide range of techniques performed on event data generated by these systems, including the visualisation, conformance checking, and enhancement of process models generating these events.
More recently, predictive process analysis techniques, often referred to as predictive process monitoring, have surfaced to support the prediction of the next event/activity in the process, the remaining process cycle time, or other goal-oriented process outcomes~\cite{DBLP:conf/bpm/Francescomarino18}.
Various predictive techniques exist, which make use of various analytics architectures, including neural networks \cite{DBLP:conf/caise/TaxVRD17}, stochastic Petri nets \cite{DBLP:conf/icsoc/Rogge-SoltiW13}, and general classification techniques \cite{DBLP:journals/tkdd/TeinemaaDRM19}.

These techniques, however, focus on a short time horizon or a well-scoped outcome in terms of prediction~\cite{DBLP:journals/tkdd/TeinemaaDRM19}.
Indeed, it is known that the next-step, or the time-to-next-step, prediction often performs poorly for long time horizons~\cite{park2020predicting}.
This, consequently, limits the range of insights that can be obtained. 
A process (model) is ever-changing. Hence, while the outcome can be predicted accurately, this might still obfuscate the underlying drivers for that outcome.
Process analysts could benefit from obtaining a more evolutionary image of the process~\cite{DBLP:conf/bpm/PollPRRR18}, including the stability or change of (parts of) the process model such as process drifts, which can inform improvement ideas~\cite{van2015pm}. 


This paper suggests a paradigm shift from the typical predictive process analysis techniques that focus on the level of individual outcome, trace, or event to techniques that focus on the entire process model.
%Current predictive techniques construct a predictive model to infer the future development of a trace, while the proposed forecasting approach infers the future development of a model which can generate future traces itself.
This is achieved via an aggregation, both at the input and output of forecasting, of directly-follows (DF) relations over all the process activity pairs as a summary of the process model over various time intervals, which are further extrapolated into the future using time series techniques modelling the number of DF occurrences over time.
Given that forecasting over long horizons results in a vast amount of information, inspired by the ideas in~\cite{DBLP:conf/er/KabicherKR11}, we propose an approach for tracking the changes in forecasted models and comparing them with AS-IS models.
The idea hence marries the concept of traditional process discovery resulting in descriptive models with a typical machine learning approach, which uses training and test sets in order to infer generalising models that can make quality predictions of the future. 

This paper makes four contributions:
\begin{enumerate}
	\item Proposes the first process model forecasting technique grounded in the time series forecasting;
	\item Evaluates the quality of the process models forecasted by our technique using standard conformance checking measures from process mining;
	\item Argues that the quality of the produced forecasts are mostly in the range of 5-15\% in terms of mean absolute percentage error in terms of conformance with actual models, which makes process model forecasting useful for practical applications; and
	\item Presents the Process Change Exploration (PCE) system for comparing AS-IS and forecasted process models.
\end{enumerate}

To this purpose, a variety of time series analysis techniques are applied to three industrial event logs using two aggregations that establish suitable time series. 
A variety of considerations regarding the use of time series analysis for event logs, and the suitability of other predictive algorithms, are discussed.
Results show that simple forecasting techniques often perform adequately without extensive parameter tuning. 

This paper is structured as follows...