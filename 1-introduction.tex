\section{Introduction}\label{sec:introduction}

The growth in the use of information systems has fuelled a wide range of data analysis techniques which intend to describe and improve their inner workings.
Process mining \cite{van2016data} is one of the strongest-growing fields in information systems analysis and encompasses the wide range of analysis which can be performed on event data generated by these systems including the visualisation, conformance checking, and improvement of process models generating these events.
More recently, predictive process analysis techniques, often referred to as predictive process monitoring, have surfaced to support the prediction of the next event/activity in the process, the remaining time, or other goal-oriented outcomes \cite{DBLP:conf/bpm/Francescomarino18}.
Various predictive techniques exist, which make use of various analytics architectures including neural networks \cite{DBLP:conf/caise/TaxVRD17}, stochastic Petri nets \cite{DBLP:conf/icsoc/Rogge-SoltiW13}, or general classification techniques \cite{DBLP:journals/tkdd/TeinemaaDRM19}.

These approaches, however, focus on a short time horizon (next-step/time-to-next-step prediction often fails to perform well over longer time horizons \cite{park2020predicting}) or a well-scoped outcome in terms of prediction \cite{DBLP:journals/tkdd/TeinemaaDRM19}.
This limits the range of insights that can be obtained as process analysts often strive to obtain a more evolutionary image of the process \cite{DBLP:conf/bpm/PollPRRR18} which captures the evolution of many factors at the same time. 
A process (model) is ever-changing and while the outcome can be predicted accurately this might still obfuscate the underlying drivers which might not be fully understood by knowing which next activities/steps are likely due in the process.

This paper envisages a paradigm shift from these typical predictive applications as its focus is not on the level of single outcomes at the level of the trace and/or event but on the model.
Current predictive techniques construct a predictive model to infer the future development of a trace, while the proposed forecasting approach infers the future development of a model which can generate future traces itself.
This is achieved by an aggregation at the model level both at the input as well as the output of the learning process by using directly-follows (DF) relations over all activity pairs as a summary of the process model at various time steps, which are further extrapolated into the future using time series techniques modelling the number of DF occurrences over time and over longer horizons.

Given that the forecasting of many relations at once over a longer horizon results in a vast amount of information, we also present how to visualise the forecasted models and show how they exist of adaptations to the current as-is model \cite{DBLP:conf/er/KabicherKR11}.

This paper makes four contributions:
\begin{enumerate}
	\item Proposes the first process model forecasting technique, which fits the the broader area of predictive process monitoring;
	\item Evaluates the quality of the forecasted process models using standard conformance checking techniques from process mining;
	\item Confirms that the achieved quality of forecasts is useful for practical applications.
	\item Provides a Process Change Exploration (PCE) system which allows process analysts to compare as-is and forecasted process models.
\end{enumerate}
To this purpose, a variety of time series analysis techniques are applied to 3 real-life event logs with two event log aggregations that establish suitable time series. 
A variety of considerations regarding the use of time series analysis for event logs are discussed, as well as the suitability of other predictive algorithms.
Results show that simple forecasting techniques often already perform adequately without extensive parameter tuning. More intricate models often fail to report consistent performance.

This paper is structured as follows...